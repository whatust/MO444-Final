\section{Conclusion \& Future Work}
\label{sec:conclusion}

Our best model was the MLP which used values from 7 days previous the one intended to be predicted,
however this model has a obvious limitation, because it cannot learn predictions and relation between visits longer than the 7 days it receives as input.
In other words no long time sequence is learned using this model only small (7-days), page and day visit relation, since both data are also inputs.
The stateless LSTM functions closely to an LSTM and end-up achieving a very similar performance to the MLP model, specially with the fully connected bypassing the LSTM layers.

The stateful LSTM takes a lot longer to train and since we could only test one architecture,
it is unclear if a model based on this component can achieve significant better results,
since our only try was unable to achieve anything.
This was a good base work to future experiments on the wikipedia page visit forecast problem.

Possible ways to improve our models are:
\begin{itemize}
	\item Use embedding to cluster pages with similar behaviors (all models);
	\item Add dropout layers to improve prediction (all models);
	\item Use visit measurements from days before 7 days prior, by calculating mean or media of a window of days (MLP model);
	\item Change hyper parameters on the layer size; (LSTM stateful model);
	\item Use a different ways to feed inputs to the model, by increasing the time-steps in each backpropagarion rollback (LSTM stateful model);
	\item increase the number of LSTM cells (LSTM stateful model);
\end{itemize}


